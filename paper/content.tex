\section{Introduction}
Efficiently identifying relations between companies based on 10k filings is an interesting and important task.
The automatic extraction of such relationships from massive data collections allows analysts to use this kind of information in their business analysis.
The FEIII Challenge 2017\footnote{\url{https://ir.nist.gov/feiii/}} addresses this problem which is located exactly at the intersection of finance and big data.
Using 10-K and 10-Q filings the challenge consists of identifying all sentences that contain the relationships between the filing financial entity and another financial entity.
In a subsequent step the resulting triples consisting of financial entity mentions, role keywords, and their contextual text should be ranked in a way, that triples which contain relevant knowledge in financial sense are ranked to top positions.
During our approach, we focus exclusively on the texts (triples) and refrain from enriching the data set with additional information such as revenue streams and other information.
In this way, we want to make sure that we focus on the core problem and not add further sources of error through the integration of additional data.
We found that one of the main difficulties is that the relatively small dataset of 1000 documents is not sufficiently large for many of the machine learning approaches we tried. 
Aggravating is that the already small amount of data is very unbalanced regarding the contained roles and the tendency to relevant and highly relevant ratings.
Thus, X, Y, Z expert ratings exist for the roles A, B, C, while for the roles G, H, J there are only X, X, and X ratings.  
In addition, there is a tendency for experts to frequently categorize documents as relevant.
Another problem stems from the fuzzy definition of ``relevance'' term.
Here it was often not clear what exactly the relevance concept refers to.

Despite these difficulties we were able to achieve an overall normalized discounted cumulative gain (NDCG) score of 98\%
% from webpage: Given a 10-K filing, an analyst asks, does Company Y, mentioned in this filing, play the role R with respect to filing Company X? Rather than simply answering "yes" or "no", a system responds by providing all the mentions of Company Y in Company X's 10-K filing, with context sentences. These triples are then in ordered by the likelihood that the context accurately defines the relationship or role R between X and Y.

\section{Description of the Data}
The dataset provided for this challenge is comprised of almost 1000 triples extracted from 25 10-K and 10-Q filings, which describe a relationship (role) between the filing company and a mentioned financial entity. Relationships are limited to ten predefined roles, namely \textit{affiliate}, \textit{agent}, \textit{counterpart}, \textit{guarantor}, \textit{insurer}, \textit{issuer}, \textit{seller}, \textit{servicer}, \textit{trustee}, and \textit{underwriter}. Additional information, such as the context a triple was extracted from and supplementary meta-data about the entities are contained as well.

Experts labelled the relevance of triples as \textit{irrelevant}, \textit{neutral}, \textit{relevant}, or \textit{highly relevant}. Depending on the context, relevance may be understood as an indicator for potential impact a relationship might have on the business of the filing company. 


\subsection{Inter Annotator Agreement}
The quality of annotations is estimated using Cohen's kappa. The inter annotator agreement (IAA) between two experts is defined as
$$
\kappa = \frac{p_0-p_e}{1-p_e},
$$
where $p_0$ is the proportion of labels with agreement, and $p_e$ the statistical chance of random agreement.

\begin{figure}
	\includegraphics[width=0.7\linewidth]{iaa}
	\caption{$\kappa$ IAA (upper triangular matrix), number of commonly rated triples (lower triangular matrix), and number of ratings (diagonal matrix)}
	\label{fig:iaa}
\end{figure}

\Fref{fig:iaa} shows the $\kappa$ for each pair of experts, as well as the number of overlapping labels. Most triplets (60\%) aren't represented in this figure, since they only received a rating by one expert and very few by more than two. The weighted average of $\overline{k}<0.5$ and few overlapping annotations are not ideal for a reliable development and evaluation of a system.

\subsection{Preparing the Dataset}
Our ranking model takes the three sentences triples are extracted from as input to predict its relevance. Especially due to the limited size of the dataset, normalising the text is an important first step, thus the strings are lower-cased and tokenised. Furthermore, tokens are lemmatised using the language model provided by SpaCy\footnote{\url{https://spacy.io/}}. Based on the normalised tokens of the entire training set, a dictionary is constructed.

As mentioned above, experts may not always agree in their rating, therefore each triple's rounded average rating is considered for all experiments. The four rating categories are mapped to numerical values (1-4).

\section{Ranking Model}
Developing a ranking model that estimates the relevance based on text assumes the presence of phrases mostly distinct for each category. Before building the dictionary, n-grams of length one to three are formed from the lemmatised tokens. A filter removes most and least frequent terms to reduce the feature space. To emphasise n-grams most meaningful for distiction, the inverse document frequency (IDF) is calculated by virtually putting all texts with the same rating in one "document". Each sentence then is encoded as a bag-of-words vector where the respective positions contain the IDF weighted by the corresponding term frequency in the sentence.

For each role, an ensemble of  linear regression models is trained to classify the sentences into the relevance categories. Problems arise due to the limited size of the annotated training set. The number of samples per role varies strongly, where sometimes not all relevance categories are represented. A model trained on such a set would not generalise properly. To circumvent that issue it was found, that training on the entire set, disregarding the roles, results in similar or even better performance.

The predictions of the model for previously unseen samples return a softmax score for each of the four categories. One could simply take the argmax as a categorical rating, but since the goal is to create a ranking, it is shown, that using the weighted sum of the confidence scores leads to better results.

\section{Evaluation and Conclusion}
The system's performance is measured by the Normalised Discounted Cumulative Gain (NDCG). The rank at position $p$ is calculated by
\begin{equation}
NDCG_p = \frac{DCG_p}{IDCG_p},\;
DCG_p = rel_1 + \sum_{i=2}^{p} \frac{rel_i}{\log_2(i+1)}
\end{equation}
where $DCG_p$ is the Discounted Cumulative Gain (DCG) when ordering items based on a given score, $IDCG_p$ the ideal DCG, and $rel_i$ the relevance of an item at position $i$ as depicted by the experts.

For the cross-evaluation, the labelled samples are split into a training and test set. Tuples in both sets do not share the filings they were extracted from and the distribution of ratings is roughly stratified. As mentioned before, eleven models are trained. One over samples ignoring the role and one for each subset restricted to a role. Predictions are only performed on test sets restricted to one role. Scores for the ranking are either continuous or categorical as described above.

\begin{table}
	\caption{Experimental results}
	\label{tab:results}
	\begin{tabular}{lcc}
		\toprule
		Approach & NDCG & $\sigma ($NDCG$)$\\
		\midrule
		Baseline (random) & 0.87 & 0.07\\
		Baseline (worst) & 0.73 & 0.13\\
		\midrule
		full set, categorical & 0.97 & 0.04 \\
		role based, categorical & 0.92 & 0.07  \\
		full set, continuous & \textbf{0.98} & 0.03 \\
		role based, continuous & 0.93 & 0.07 \\
		\bottomrule
	\end{tabular}
\end{table}

In \fref{tab:results} lists the mean NDCG scores and the standard deviation ($\sigma$). For comparison we take a baseline of the worst possible ranking (iverse order of the ideal ranking) and the average of multiple random rankings. This shows, that disregarding roles during training significantly improves the performance and continuous scoring helps to correct the score in cases of slight uncertainty.

\begin{figure}
	\begin{subfigure}[t]{0.5\linewidth}
		\centering
		\includegraphics[width=\linewidth]{conf_full}
		\caption{Trained on all samples}
	\end{subfigure}%
	~ 
	\begin{subfigure}[t]{0.5\linewidth}
		\includegraphics[width=\linewidth]{conf_role}
		\caption{Trained on role samples}
	\end{subfigure}
	\caption{Normalised aggregated confusion matrices for the model with different training sets}
	\label{fig:confmatrix}
\end{figure}

It is interesting to note, that evaluating the classification model alone, the F1-Score is only around 0.49 ($\sigma=$0.17) when training on small subsets and 0.73 ($\sigma=$0.27) for the full set. \Fref{fig:confmatrix} clearly shows the difficulties, which appear to be a result of the bias towards relevant documents due to the imbalanced set.

Training a model on text usually requires a reasonably large corpus. With the data at hand we had to pay close attention to the selection of features, since seemingly very specific terms are used, which likely negatively affacts the ability of the model to generalise to unseen samples.

%
%\begin{table}
%	\caption{Experimental results}
%	\label{tab:freq}
%	\begin{tabular}{lcccc}
%		\toprule
%		Approach & NDCG & $\sigma ($NDCG$)$ & F1-score & $\sigma($F1-Score$)$\\
%		\midrule
%		Baseline (random) & $0.87$ & $0.07$ & - & - \\
%		Baseline (worst) & $0.73$ & $0.13$ & - & - \\
%		full set, categorical & $0.97$ & $0.04$ & $0.73$ & $0.27$ \\
%		role based, categorical & $0.92$ & $0.07$ & $0.49$ & $0.17$ \\
%		full set, continuous & $0.98$ & $0.03$ & - & - \\
%		role based, continuous & $0.93$ & $0.07$ & - & - \\
%		\bottomrule
%	\end{tabular}
%\end{table}
%
